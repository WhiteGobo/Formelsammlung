\chapter{Mechanik}
  \section{Newton-Mechanik}
  \section{Langrange-Mechanik}
    Newtonmechanik erweitert um Zwangskräfte $Z_i$:
    \begin{equation}
      m \ableitungii{t} x_i = F_i + Z_i
    \end{equation}

    D'Alembertsches Prinzip:
    virtuelle Verrückung $\delta r_j$ \\
    1) zeitlos, $\delta t=0$ \\
    2) verträglich mit Zwangsbedingungen \\
    3) infinitesimal klein 

    \begin{align*}
      \sum^N \left( m \ableitung{t} x_i -F_i\right) \delta r_i = \sum^N Z_i \delta r_i &\\
      \sum^N Z_i \delta r_i= 0 &\\
      \sum^N m \left(\ableitungii{t} x_i -F_i \right) \delta r_i= 0 &\\
      \delta r_i= \sum^{3N-k} \frac{dr_j}{dq_j} \delta q_j&, \mathrm{~mit ~} r_i=r_i(q_1, ... , q_{3N-k}) \\
      \sum_i^N D_i \delta r_i = \sum_j^{3N-k} \left( \sum_i^N F_i \ableitung{q_j} r_j \right) \delta q_j =  \sum_j^{3N-k} Q_j \delta q_j &
      \end{align*}
      \begin{align*}
      \sum_i^N m \ableitungii{t} r_i \delta r_i &= \sum_j^{3N-k} \left( \sum_i^N m_i \ableitungii{t} r_i \ableitung{q_j} r_i \right) \delta q_j \\
      &= \sum_j^{3N-k} \left( \sum_i^N \ableitung{t} \left( m_i \ableitung{t} r_i \ableitung{q_j} r_i \right) - m_i \ableitung{t} r_i \ableitung{t} \ableitung{q_j} r_i \right) \delta q_j = \\
      &= \sum_j^{3N-k} \left( \sum_i^N \ableitung{t} \left( m_i v_i \ableitung{\dot q_j} v_i \right) - m_i v_i \ableitung{q_j} v_i \right) \delta q_j= \\
      &= \sum_j^{3N-k} \left( \ableitung{t} \left( \ableitung{\dot q_j} \sum_i^N \frac{1}{2} m_i v_i^2 \right) - \ableitung{q_j} \sum_i^N \frac{1}{2} m_i v_i^2 \right) \delta q_j = \\
      &= \sum_j^{3N-k} \left( \ableitung{t} \ableitung{\dot q_j} T - \ableitung{q_j} T \right) \delta q_j \\
    \end{align*}
    \begin{equation}
      \sum_j^{3N-k} \left( \ableitung{t} \ableitungv{\dot q_j}{T} - \ableitungv{q_j}{T} - Q_j \right) \delta q_j= 0 
    \end{equation}
      ,mit $ T=\frac{1}{2} \sum_i^N m_i v_i^2$ \\
      ,und $Q = \sum_i^N F_i \ableitung{q_j} r_i$
    \begin{equation}
      Q_j = \sum_i^N F_i \ableitung{q_j} r_i
      = -\sum_i^N \ableitungv{r_i}{V} \ableitungv{q_j}{r_i} 
      = -\sum_i^N \ableitungv{q_j}{V}
    \end{equation}
    Langrange-Funktion:
    \begin{gather}
      L =T-V \\
      \ableitung{t} \ableitungv{\dot q_j}{L} - \ableitungv{q_j}{L} = 0
    \end{gather}
Vorgehensweise: \\
    1) generalisierte Koordinaten finden, mit $r=r\left(q_1, ..., q_{3N-k}\right)$ \\
    2) T, V aufstellen in kartesischen Koordinaten \\
    3) T, V aufstellen in generalisierten Koordinaten \\
    4) $L=T-V$ \\
    5) Langrange-Gleichung aufstellen \\
    6) lösen: $q_1(t) =...; q_2(t) =...$
