\chapter{Molekülphysik}
  \section{Atome}
  \subsection{Quantenzahlen}
    \minisec{Hauptquantenzahl}
    \minisec{Nebenquantenzahl}
      \begin{equation}
        l=0, ... n
      \end{equation}
    \minisec{Drehimpulsquantenzahl}
      \begin{equation}
        m_l=-l, ... , +l
      \end{equation}
    \minisec{Spinquantenzahl}
      \begin{equation}
        s=\pm \frac{1}{2}
      \end{equation}
  \subsection{Feinstruktur}
    \begin{equation}
      \Delta E_{Fs} = E_1 \left( \frac{Z^2\alpha^2}{n^2} \left( \frac{3}{4} - \frac{n}{j+\frac{1}{2}} \right) \right)
    \end{equation}
  \subsection{Zeeman-Effekt}
    \minisec{Nutzen}
      Durch den Zeeman-Effekt wurden hohe Magnetfelder in Sonnenflecken nachgewiesen.\\
      Er wird bei Atomspektroskopie benutzt. \\
      Eine Theorie über die Magnetrezeption über Tiere geht davon aus, dass ein Protein in der Netzhaut von Vögeln den Magnetsinn über den Zeeman-Effekt hervorruft. 
    \minisec{normaler Zeemann-Effekt}
      \begin{equation}
        E_{mag}=-\left( \mu \cdot B \right) = -\mu_z B = -\gamma  l_z B
      \end{equation}
      mit der zugehörigen gyromagnetische Konstante $\gamma = \frac{e}{2m_e}$

    \minisec{anormaler Zeeman-Effekt}
      \begin{equation}
        E_{mag}=- g_j m_j \hbar  \gamma B
      \end{equation}
      mit dem g-Faktor $g_j$ aus der Land\'{e}-Formel, \\
      der gyromagnetischen Konstante $\gamma = \frac{e}{2m_e}$, \\
      und der magnetischen Quantenzahl des Drehimpulses.
      
    \minisec{Paschen-Back-Effekt}
      Bei hoher Feldstärke wird entsteht der Paschen-Back-Effekt.
      \begin{equation}
        E_{mag}= \gamma  \hbar  B \left( g_l m_l + g_s m_s \right)
      \end{equation}
      mit der gyromagnetischen Konstante $\gamma = \frac{e}{2m_e}$, \\
      mit $g_l=1$ , der magnetischen Quantenzhal $m_l$, \\
      $g_s=\frac{1}{2}$, und der magnetisches Spinquantenzahl $m_s$.
      \\
      Man kann den Energieunterschied so berechnen:
      \begin{equation}
        \Delta E = \mu_b B \left( m_l + 2 m_s \right)
      \end{equation}

  \subsection{Boltzmann Verteilung}
    Die Wahrscheinlichkeit $p_j$ ein Teilchen in einem bestimmten Zustand mit der jeweils zugehörigen Energie $E_j$:
    \begin{equation}
      p_j= \frac{1}{Z} g_j \exp{-\beta E_j}
    \end{equation}
    mit dem Entartungsgrad $g_j$ des Zustands j, also der Anzahl der Zustände gleicher Energie $E_j$, \\
    der Normierung $Z$, auch Zustandssumme genannt, \\
    und der Energienormierung $\beta = \frac{1}{k_b T}$, mit $k_b$, der Boltzmann-Konstante und $T$ der Temperatur.
  \subsection{Fermi-Dirac-Verteilung}
    Die Wahrscheinlichkeit ein Teilchen sog. Fermionen in der Quantenmechanik in einem bestimmten Zustand vorzufinden. 
    \begin{equation}
      p_{(E)} = \frac{1}{\exp{\frac{E-\mu }{k_b T}} +1}
    \end{equation}
    mit dem chemischen Potential $\mu$ wobei bei $T=0$ gilt $\mu = E_f$ wobei $E_f$ als Fermi-Energie bezeichnt wird,\\
    der Boltzmann-Konstante $k_b$, und $T$ der Temperatur.


    \section{Halbleiter}
      \subsection{Definition}
        Halbleiter sind alle nicht leitenden dotierbaren Atome/Stoffe.


    \section{Isolator}


    \section{Metalle}
