\chapter{Mathematische Methoden}
\section{Vektorräume}
\subsection{Gram-Schmidtsches Orthogonalisierungsverfahren}
Verfahren um eine Basis $\left( v_1, .., v_n\right)$ zu orthogonalisieren.
\begin{gather}
a_1=v_1 \\
    a_n=v_n - \sum_{i=1}^{n-1} \frac{a_i*v_n}{a_i*a_i}a_i 
    \end{gather}

    \section{Quaternion}
    \subsection{Definition}
    Eine Quaternion ist $4$-dimensional.:
    \begin{equation}
    q=q_0+q_1\cdot i+q_2\cdot j+q_3\cdot k
    \end{equation}
    Dabei gilt für die imginären Anteile:
    \begin{gather*}
    i^2=j^2=k^2=-1\\
	i\cdot j=k\\
	j\cdot k=i\\
	k \cdot i=j\\
	j \cdot i= -k\\
	\dots
	\end{gather*}

	\subsection{Drehung von Vektoren}
	\begin{equation}
	\rho_q:x\rightarrow qxq^-1
	\end{equation}

	\begin{equation}
	q=\cos{\frac{\alpha}{2}} + \epsilon \sin{\frac{\alpha}{2}}
	\end{equation}
	\section{Matrizen}
	\subsection{Eigenwerte}
	\subsection{Cayley-Hamilton}
Satz: Die Matrix $A$ ist eine Nullstelle ihres charakteristischen Polynoms\\
		Beispiel: P: $\lambda^2 + \lambda = 0$\\
		Dann gilt: $A^2 + A = 0$


\subsection{Maßeinheiten}
	\begin{figure}[h]
		\begin{tabular}{ccl}
			Y & Yotta     & $10^{24}$ \\
			Z & Zetta     & $10^{21}$ \\
			E & Exa       & $10^{18}$ \\
			P & Peta      & $10^{15}$ \\
			T & Tera      & $10^{12}$ \\
			G & Giga      & $10^{9}$ \\
			M & Mega      & $10^{6}$ \\
			k & Kilo      & $10^{3}$ \\
			m & Milli     & $10^{-3}$ \\
			$\mu$ & Mikro & $10^{-6}$ \\
			n & Nano      & $10^{-9}$ \\
			p & Piko      & $10^{-12}$ \\
			f & Femto     & $10^{-15}$ \\
			a & Atto      & $10^{-18}$ \\
			z & Zepto     & $10^{-21}$ \\
			y & Yokto     & $10^{-24}$ \\
		\end{tabular}
		\caption{SI-Präfixe}
	\end{figure}
