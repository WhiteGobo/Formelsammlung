\documentclass{scrbook}

\usepackage[ngerman]{babel}
\usepackage[utf8]{inputenc}
\usepackage[fleqn]{mathtools}
\usepackage{amsmath}
\usepackage{amssymb}
\usepackage[exponent-product=\cdot]{siunitx}
\usepackage{verbatim}


% paket fürbilder
\usepackage{tikz}
\usetikzlibrary{shapes,calc}

%% Metadaten
\usepackage[pdftitle={Formelsammlung},
    pdfsubject={Formeln und Werte der Physik},
    pdfauthor={Richard Fechner},
    pdfkeywords={Formel, Physik, Mathematik }]
    {hyperref}

%% Daten zu Autor
\title{Formeln und Werte der Physik}
\author{Richard Fechner}
\date{\today}

%% eigene Befehle
\newcommand{\ableitung}[1]{\frac{\mathrm{d}}{\mathrm{d} #1}}
\newcommand{\ableitungv}[2]{\frac{\mathrm{d}#2}{\mathrm{d} #1}}
\newcommand{\ableitungii}[1]{\frac{\mathrm{d}^2}{\mathrm{d} #1^2}}
\renewcommand{\exp}[1]{exp \left( #1 \right)}

%% bestimme welche Kapitel in formelsamlung augenomen werden
\includeonly{
  molekuelphysik/molekuelphysik
  ,mathematik/mathematik
  ,mechanik/mechanik
  ,naturkonstanten/naturkonstanten
  ,chemie/chemie
}

%% Stelle Nummeroerung auf (kapitel).(gleichung)
\renewcommand{\theequation}{\arabic{section}.\arabic{equation}}

%% Anfang des Dokuments
\begin{document}

\maketitle

\tableofcontents
\clearpage

\chapter{Mathematische Methoden}
\section{Vektorräume}
\subsection{Gram-Schmidtsches Orthogonalisierungsverfahren}
Verfahren um eine Basis $\left( v_1, .., v_n\right)$ zu orthogonalisieren.
\begin{gather}
a_1=v_1 \\
    a_n=v_n - \sum_{i=1}^{n-1} \frac{a_i*v_n}{a_i*a_i}a_i 
    \end{gather}

    \section{Quaternion}
    \subsection{Definition}
    Eine Quaternion ist $4$-dimensional.:
    \begin{equation}
    q=q_0+q_1\cdot i+q_2\cdot j+q_3\cdot k
    \end{equation}
    Dabei gilt für die imginären Anteile:
    \begin{gather*}
    i^2=j^2=k^2=-1\\
	i\cdot j=k\\
	j\cdot k=i\\
	k \cdot i=j\\
	j \cdot i= -k\\
	\dots
	\end{gather*}

	\subsection{Drehung von Vektoren}
	\begin{equation}
	\rho_q:x\rightarrow qxq^-1
	\end{equation}

	\begin{equation}
	q=\cos{\frac{\alpha}{2}} + \epsilon \sin{\frac{\alpha}{2}}
	\end{equation}
	\section{Matrizen}
	\subsection{Eigenwerte}
	\subsection{Cayley-Hamilton}
Satz: Die Matrix $A$ ist eine Nullstelle ihres charakteristischen Polynoms\\
		Beispiel: P: $\lambda^2 + \lambda = 0$\\
		Dann gilt: $A^2 + A = 0$


\subsection{Maßeinheiten}
	\begin{figure}[h]
		\begin{tabular}{ccl}
			Y & Yotta     & $10^{24}$ \\
			Z & Zetta     & $10^{21}$ \\
			E & Exa       & $10^{18}$ \\
			P & Peta      & $10^{15}$ \\
			T & Tera      & $10^{12}$ \\
			G & Giga      & $10^{9}$ \\
			M & Mega      & $10^{6}$ \\
			k & Kilo      & $10^{3}$ \\
			m & Milli     & $10^{-3}$ \\
			$\mu$ & Mikro & $10^{-6}$ \\
			n & Nano      & $10^{-9}$ \\
			p & Piko      & $10^{-12}$ \\
			f & Femto     & $10^{-15}$ \\
			a & Atto      & $10^{-18}$ \\
			z & Zepto     & $10^{-21}$ \\
			y & Yokto     & $10^{-24}$ \\
		\end{tabular}
		\caption{SI-Präfixe}
	\end{figure}


\chapter{Mechanik}
  \section{Newton-Mechanik}
  \section{Langrange-Mechanik}
    Newtonmechanik erweitert um Zwangskräfte $Z_i$:
    \begin{equation}
      m \ableitungii{t} x_i = F_i + Z_i
    \end{equation}

    D'Alembertsches Prinzip:
    virtuelle Verrückung $\delta r_j$ \\
    1) zeitlos, $\delta t=0$ \\
    2) verträglich mit Zwangsbedingungen \\
    3) infinitesimal klein 

    \begin{align*}
      \sum^N \left( m \ableitung{t} x_i -F_i\right) \delta r_i = \sum^N Z_i \delta r_i &\\
      \sum^N Z_i \delta r_i= 0 &\\
      \sum^N m \left(\ableitungii{t} x_i -F_i \right) \delta r_i= 0 &\\
      \delta r_i= \sum^{3N-k} \frac{dr_j}{dq_j} \delta q_j&, \mathrm{~mit ~} r_i=r_i(q_1, ... , q_{3N-k}) \\
      \sum_i^N D_i \delta r_i = \sum_j^{3N-k} \left( \sum_i^N F_i \ableitung{q_j} r_j \right) \delta q_j =  \sum_j^{3N-k} Q_j \delta q_j &
      \end{align*}
      \begin{align*}
      \sum_i^N m \ableitungii{t} r_i \delta r_i &= \sum_j^{3N-k} \left( \sum_i^N m_i \ableitungii{t} r_i \ableitung{q_j} r_i \right) \delta q_j \\
      &= \sum_j^{3N-k} \left( \sum_i^N \ableitung{t} \left( m_i \ableitung{t} r_i \ableitung{q_j} r_i \right) - m_i \ableitung{t} r_i \ableitung{t} \ableitung{q_j} r_i \right) \delta q_j = \\
      &= \sum_j^{3N-k} \left( \sum_i^N \ableitung{t} \left( m_i v_i \ableitung{\dot q_j} v_i \right) - m_i v_i \ableitung{q_j} v_i \right) \delta q_j= \\
      &= \sum_j^{3N-k} \left( \ableitung{t} \left( \ableitung{\dot q_j} \sum_i^N \frac{1}{2} m_i v_i^2 \right) - \ableitung{q_j} \sum_i^N \frac{1}{2} m_i v_i^2 \right) \delta q_j = \\
      &= \sum_j^{3N-k} \left( \ableitung{t} \ableitung{\dot q_j} T - \ableitung{q_j} T \right) \delta q_j \\
    \end{align*}
    \begin{equation}
      \sum_j^{3N-k} \left( \ableitung{t} \ableitungv{\dot q_j}{T} - \ableitungv{q_j}{T} - Q_j \right) \delta q_j= 0 
    \end{equation}
      ,mit $ T=\frac{1}{2} \sum_i^N m_i v_i^2$ \\
      ,und $Q = \sum_i^N F_i \ableitung{q_j} r_i$
    \begin{equation}
      Q_j = \sum_i^N F_i \ableitung{q_j} r_i
      = -\sum_i^N \ableitungv{r_i}{V} \ableitungv{q_j}{r_i} 
      = -\sum_i^N \ableitungv{q_j}{V}
    \end{equation}
    Langrange-Funktion:
    \begin{gather}
      L =T-V \\
      \ableitung{t} \ableitungv{\dot q_j}{L} - \ableitungv{q_j}{L} = 0
    \end{gather}
Vorgehensweise: \\
    1) generalisierte Koordinaten finden, mit $r=r\left(q_1, ..., q_{3N-k}\right)$ \\
    2) T, V aufstellen in kartesischen Koordinaten \\
    3) T, V aufstellen in generalisierten Koordinaten \\
    4) $L=T-V$ \\
    5) Langrange-Gleichung aufstellen \\
    6) lösen: $q_1(t) =...; q_2(t) =...$


\chapter{Molekülphysik}
  \section{Atome}
  \subsection{Quantenzahlen}
    \minisec{Hauptquantenzahl}
    \minisec{Nebenquantenzahl}
      \begin{equation}
        l=0, ... n
      \end{equation}
    \minisec{Drehimpulsquantenzahl}
      \begin{equation}
        m_l=-l, ... , +l
      \end{equation}
    \minisec{Spinquantenzahl}
      \begin{equation}
        s=\pm \frac{1}{2}
      \end{equation}
  \subsection{Feinstruktur}
    \begin{equation}
      \Delta E_{Fs} = E_1 \left( \frac{Z^2\alpha^2}{n^2} \left( \frac{3}{4} - \frac{n}{j+\frac{1}{2}} \right) \right)
    \end{equation}
  \subsection{Zeeman-Effekt}
    \minisec{Nutzen}
      Durch den Zeeman-Effekt wurden hohe Magnetfelder in Sonnenflecken nachgewiesen.\\
      Er wird bei Atomspektroskopie benutzt. \\
      Eine Theorie über die Magnetrezeption über Tiere geht davon aus, dass ein Protein in der Netzhaut von Vögeln den Magnetsinn über den Zeeman-Effekt hervorruft. 
    \minisec{normaler Zeemann-Effekt}
      \begin{equation}
        E_{mag}=-\left( \mu \cdot B \right) = -\mu_z B = -\gamma  l_z B
      \end{equation}
      mit der zugehörigen gyromagnetische Konstante $\gamma = \frac{e}{2m_e}$

    \minisec{anormaler Zeeman-Effekt}
      \begin{equation}
        E_{mag}=- g_j m_j \hbar  \gamma B
      \end{equation}
      mit dem g-Faktor $g_j$ aus der Land\'{e}-Formel, \\
      der gyromagnetischen Konstante $\gamma = \frac{e}{2m_e}$, \\
      und der magnetischen Quantenzahl des Drehimpulses.
      
    \minisec{Paschen-Back-Effekt}
      Bei hoher Feldstärke wird entsteht der Paschen-Back-Effekt.
      \begin{equation}
        E_{mag}= \gamma  \hbar  B \left( g_l m_l + g_s m_s \right)
      \end{equation}
      mit der gyromagnetischen Konstante $\gamma = \frac{e}{2m_e}$, \\
      mit $g_l=1$ , der magnetischen Quantenzhal $m_l$, \\
      $g_s=\frac{1}{2}$, und der magnetisches Spinquantenzahl $m_s$.
      \\
      Man kann den Energieunterschied so berechnen:
      \begin{equation}
        \Delta E = \mu_b B \left( m_l + 2 m_s \right)
      \end{equation}

  \subsection{Boltzmann Verteilung}
    Die Wahrscheinlichkeit $p_j$ ein Teilchen in einem bestimmten Zustand mit der jeweils zugehörigen Energie $E_j$:
    \begin{equation}
      p_j= \frac{1}{Z} g_j \exp{-\beta E_j}
    \end{equation}
    mit dem Entartungsgrad $g_j$ des Zustands j, also der Anzahl der Zustände gleicher Energie $E_j$, \\
    der Normierung $Z$, auch Zustandssumme genannt, \\
    und der Energienormierung $\beta = \frac{1}{k_b T}$, mit $k_b$, der Boltzmann-Konstante und $T$ der Temperatur.
  \subsection{Fermi-Dirac-Verteilung}
    Die Wahrscheinlichkeit ein Teilchen sog. Fermionen in der Quantenmechanik in einem bestimmten Zustand vorzufinden. 
    \begin{equation}
      p_{(E)} = \frac{1}{\exp{\frac{E-\mu }{k_b T}} +1}
    \end{equation}
    mit dem chemischen Potential $\mu$ wobei bei $T=0$ gilt $\mu = E_f$ wobei $E_f$ als Fermi-Energie bezeichnt wird,\\
    der Boltzmann-Konstante $k_b$, und $T$ der Temperatur.


    \section{Halbleiter}
      \subsection{Definition}
        Halbleiter sind alle nicht leitenden dotierbaren Atome/Stoffe.


    \section{Isolator}


    \section{Metalle}


\chapter{Chemie}
\section{Atome}

\newpage
\subsection{Periodensystem von Mendeleev}

%% Periodensystem in tikz zeichnen%%%%%%%%%%%%%%%%%%%%%%%%
%%%%%%%%%%%%%%%%%%%%%%%%%%%%%%%%%%%%%%%%%%%%%%%%%%%%%%%%%%%%
\newcommand{\CommonElementTextFormat}[5]
{
	\begin{minipage}{2.2cm}
	\centering
	{\textbf{#1} \hfill #2}%
	\linebreak \linebreak
	{\textbf{#3}}%
	\linebreak
	{{#4}} %  \\%
	\linebreak
	{{#5} \hfill { }}
	\end{minipage}
}

\newcommand{\NaturalElementTextFormat}[5]
{
	\CommonElementTextFormat{#1}{#2}{\LARGE {#3}}{#4}{#5}
}

\newcommand{\OutlineText}[1]
{
	%
		% Couldn't find a nicer way of doing an outline font with TikZ
		% other than using pdfliteral 1 Tr
		\pdfliteral direct {0.5 w 1 Tr}{#1}%
		\pdfliteral direct {1 w 0 Tr}%
}

\newcommand{\SyntheticElementTextFormat}[5]
{
	\CommonElementTextFormat{#1}{#2}{\OutlineText{\LARGE #3}}{#4}{#5}
}

\begin{tikzpicture}[font=\sffamily, scale=0.45, transform shape]

%% Fill Color Styles
\tikzstyle{ElementFill} = [fill=yellow!15]
\tikzstyle{AlkaliMetalFill} = [fill=blue!55]
\tikzstyle{AlkalineEarthMetalFill} = [fill=blue!40]
\tikzstyle{MetalFill} = [fill=blue!25]
\tikzstyle{MetalloidFill} = [fill=orange!25]
\tikzstyle{NonmetalFill} = [fill=green!25]
\tikzstyle{HalogenFill} = [fill=green!40]
\tikzstyle{NobleGasFill} = [fill=green!55]
\tikzstyle{LanthanideActinideFill} = [fill=purple!25]

%% Element Styles
\tikzstyle{Element} = [draw=black, ElementFill,
	minimum width=2.75cm, minimum height=2.75cm, node distance=2.75cm]
	\tikzstyle{AlkaliMetal} = [Element, AlkaliMetalFill]
	\tikzstyle{AlkalineEarthMetal} = [Element, AlkalineEarthMetalFill]
	\tikzstyle{Metal} = [Element, MetalFill]
	\tikzstyle{Metalloid} = [Element, MetalloidFill]
	\tikzstyle{Nonmetal} = [Element, NonmetalFill]
	\tikzstyle{Halogen} = [Element, HalogenFill]
	\tikzstyle{NobleGas} = [Element, NobleGasFill]
	\tikzstyle{LanthanideActinide} = [Element, LanthanideActinideFill]
	\tikzstyle{PeriodLabel} = [font={\sffamily\LARGE}, node distance=2.0cm]
	\tikzstyle{GroupLabel} = [font={\sffamily\LARGE}, minimum width=2.75cm, node distance=2.0cm]
	\tikzstyle{TitleLabel} = [font={\sffamily\Huge\bfseries}]

	%% Group 1 - IA
	\node[name=H, Element] {\NaturalElementTextFormat{2}{1.0079}{H}{Hydrogen}{}};
\node[name=Li, below of=H, AlkaliMetal] {\NaturalElementTextFormat{3}{6.941}{Li}{Lithium}{}};
\node[name=Na, below of=Li, AlkaliMetal] {\NaturalElementTextFormat{11}{22.990}{Na}{Sodium}{}};
\node[name=K, below of=Na, AlkaliMetal] {\NaturalElementTextFormat{19}{39.098}{K}{Potassium}{}};
\node[name=Rb, below of=K, AlkaliMetal] {\NaturalElementTextFormat{37}{85.468}{Rb}{Rubidium}{}};
\node[name=Cs, below of=Rb, AlkaliMetal] {\NaturalElementTextFormat{55}{132.91}{Cs}{Caesium}{}};
\node[name=Fr, below of=Cs, AlkaliMetal] {\NaturalElementTextFormat{87}{223}{Fr}{Francium}{}};

%% Group 2 - IIA
\node[name=Be, right of=Li, AlkalineEarthMetal] {\NaturalElementTextFormat{4}{9.0122}{Be}{Beryllium}{}};
\node[name=Mg, below of=Be, AlkalineEarthMetal] {\NaturalElementTextFormat{12}{24.305}{Mg}{Magnesium}{}};
\node[name=Ca, below of=Mg, AlkalineEarthMetal] {\NaturalElementTextFormat{20}{40.078}{Ca}{Calcium}{}};
\node[name=Sr, below of=Ca, AlkalineEarthMetal] {\NaturalElementTextFormat{38}{87.62}{Sr}{Strontium}{}};
\node[name=Ba, below of=Sr, AlkalineEarthMetal] {\NaturalElementTextFormat{56}{137.33}{Ba}{Barium}{}};
\node[name=Ra, below of=Ba, AlkalineEarthMetal] {\NaturalElementTextFormat{88}{226}{Ra}{Radium}{}};

%% Group 3-12
\node[name=ScZn, right of=Ca, Metal] {\NaturalElementTextFormat{21-30}{}{Sc-Zn}{}{}};
\node[name=YCd, below of=ScZn, Metal] {\NaturalElementTextFormat{39-48}{}{Y-Cd}{}{}};
\node[name=LaHg, below of=YCd, Metal, left color=purple!25, right color=blue!25] {\NaturalElementTextFormat{57-80}{}{La-Hg}{}{}};
\node[name=AcUub, below of=LaHg, Metal, left color=purple!25, right color=blue!25] {\NaturalElementTextFormat{89-112}{}{Ac-Uub}{}{}};

%% Group 13 - IIIA
\node[name=Ga, right of=ScZn, Metal] {\NaturalElementTextFormat{31}{69.723}{Ga}{Gallium}{}};
\node[name=Al, above of=Ga, Metal] {\NaturalElementTextFormat{13}{26.982}{Al}{Aluminium}{}};
\node[name=B, above of=Al, Metalloid] {\NaturalElementTextFormat{5}{10.811}{B}{Boron}{}};
\node[name=In, below of=Ga, Metal] {\NaturalElementTextFormat{49}{114.82}{In}{Indium}{}};
\node[name=Tl, below of=In, Metal] {\NaturalElementTextFormat{81}{204.38}{Tl}{Thallium}{}};
\node[name=Uut, below of=Tl, Metal] {\SyntheticElementTextFormat{113}{284}{Uut}{Ununtrium}{}};

%% Group 14 - IVA
\node[name=C, right of=B, Nonmetal] {\NaturalElementTextFormat{6}{12.011}{C}{Carbon}{}};
\node[name=Si, below of=C, Metalloid] {\NaturalElementTextFormat{14}{28.086}{Si}{Silicon}{}};
\node[name=Ge, below of=Si, Metalloid] {\NaturalElementTextFormat{32}{72.64}{Ge}{Germanium}{}};
\node[name=Sn, below of=Ge, Metal] {\NaturalElementTextFormat{50}{118.71}{Sn}{Tin}{}};
\node[name=Pb, below of=Sn, Metal] {\NaturalElementTextFormat{82}{207.2}{Pb}{Lead}{}};
\node[name=Uuq, below of=Pb, Metal] {\SyntheticElementTextFormat{114}{289}{Uuq}{Ununquadium}{}};

%% Group 15 - VA
\node[name=N, right of=C, Nonmetal] {\NaturalElementTextFormat{7}{14.007}{N}{Nitrogen}{}};
\node[name=P, below of=N, Nonmetal] {\NaturalElementTextFormat{15}{30.974}{P}{Phosphorus}{}};
\node[name=As, below of=P, Metalloid] {\NaturalElementTextFormat{33}{74.922}{As}{Arsenic}{}};
\node[name=Sb, below of=As, Metalloid] {\NaturalElementTextFormat{51}{121.76}{Sb}{Antimony}{}};
\node[name=Bi, below of=Sb, Metal] {\NaturalElementTextFormat{83}{208.98}{Bi}{Bismuth}{}};
\node[name=Uup, below of=Bi, Metal] {\SyntheticElementTextFormat{115}{288}{Uup}{Ununpentium}{}};

%% Group 16 - VIA
\node[name=O, right of=N, Nonmetal] {\NaturalElementTextFormat{8}{15.999}{O}{Oxygen}{}};
\node[name=S, below of=O, Nonmetal] {\NaturalElementTextFormat{16}{32.065}{S}{Sulphur}{}};
\node[name=Se, below of=S, Nonmetal] {\NaturalElementTextFormat{34}{78.96}{Se}{Selenium}{}};
\node[name=Te, below of=Se, Metalloid] {\NaturalElementTextFormat{52}{127.6}{Te}{Tellurium}{}};
\node[name=Po, below of=Te, Metalloid] {\NaturalElementTextFormat{84}{209}{Po}{Polonium}{}};
\node[name=Uuh, below of=Po, Metal] {\SyntheticElementTextFormat{116}{293}{Uuh}{Ununhexium}{}};

%% Group 17 - VIIA
\node[name=F, right of=O, Halogen] {\NaturalElementTextFormat{9}{18.998}{F}{Flourine}{}};
\node[name=Cl, below of=F, Halogen] {\NaturalElementTextFormat{17}{35.453}{Cl}{Chlorine}{}};
\node[name=Br, below of=Cl, Halogen] {\NaturalElementTextFormat{35}{79.904}{Br}{Bromine}{}};
\node[name=I, below of=Br, Halogen] {\NaturalElementTextFormat{53}{126.9}{I}{Iodine}{}};
\node[name=At, below of=I, Halogen] {\NaturalElementTextFormat{85}{210}{At}{Astatine}{}};
\node[name=Uus, below of=At, Element] {\SyntheticElementTextFormat{117}{292}{Uus}{Ununseptium}{}}; 

%% Group 18 - VIIIA
\node[name=Ne, right of=F, NobleGas] {\NaturalElementTextFormat{10}{20.180}{Ne}{Neon}{}};
\node[name=He, above of=Ne, NobleGas] {\NaturalElementTextFormat{2}{4.0025}{He}{Helium}{}};
\node[name=Ar, below of=Ne, NobleGas] {\NaturalElementTextFormat{18}{39.948}{Ar}{Argon}{}};
\node[name=Kr, below of=Ar, NobleGas] {\NaturalElementTextFormat{36}{83.8}{Kr}{Krypton}{}};
\node[name=Xe, below of=Kr, NobleGas] {\NaturalElementTextFormat{54}{131.29}{Xe}{Xenon}{}};
\node[name=Rn, below of=Xe, NobleGas] {\NaturalElementTextFormat{86}{222}{Rn}{Radon}{}};
\node[name=Uuo, below of=Rn, Nonmetal] {\SyntheticElementTextFormat{118}{294}{Uuo}{Ununoctium}{}}; 

%% Group 3-12
%% Group 3 - IIIB
\node[name=Sc, below of=Fr, Metal, yshift=-1.5cm] {\NaturalElementTextFormat{21}{44.956}{Sc}{Scandium}{}};
\node[name=Y, below of=Sc, Metal] {\NaturalElementTextFormat{39}{88.906}{Y}{Yttrium}{}};
\node[name=LaLu, below of=Y, LanthanideActinide] {\NaturalElementTextFormat{57-71}{}{La-Lu}{Lanthanide}{}};
\node[name=AcLr, below of=LaLu, LanthanideActinide] {\NaturalElementTextFormat{89-103}{}{Ac-Lr}{Actinide}{}};

%% Group 4 - IVB
\node[name=Ti, right of=Sc, Metal] {\NaturalElementTextFormat{22}{47.867}{Ti}{Titanium}{}};
\node[name=Zr, below of=Ti, Metal] {\NaturalElementTextFormat{40}{91.224}{Zr}{Zirconium}{}};
\node[name=Hf, below of=Zr, Metal] {\NaturalElementTextFormat{72}{178.49}{Hf}{Halfnium}{}};
\node[name=Rf, below of=Hf, Metal] {\SyntheticElementTextFormat{104}{261}{Rf}{Rutherfordium}{}};

%% Group 5 - VB
\node[name=V, right of=Ti, Metal] {\NaturalElementTextFormat{23}{50.942}{V}{Vanadium}{}};
\node[name=Nb, below of=V, Metal] {\NaturalElementTextFormat{41}{92.906}{Nb}{Niobium}{$4d^45s^1$}};
\node[name=Ta, below of=Nb, Metal] {\NaturalElementTextFormat{73}{180.95}{Ta}{Tantalum}{}};
\node[name=Db, below of=Ta, Metal] {\SyntheticElementTextFormat{105}{262}{Db}{Dubnium}{}};

%% Group 6 - VIB
\node[name=Cr, right of=V, Metal] {\NaturalElementTextFormat{24}{51.996}{Cr}{Chromium}{$3d^54s^1$}};
\node[name=Mo, below of=Cr, Metal] {\NaturalElementTextFormat{42}{95.94}{Mo}{Molybdenum}{$4d^5s^1$}};
\node[name=W, below of=Mo, Metal] {\NaturalElementTextFormat{74}{183.84}{W}{Tungsten}{}};
\node[name=Sg, below of=W, Metal] {\SyntheticElementTextFormat{106}{266}{Sg}{Seaborgium}{}};

%% Group 7 - VIIB
\node[name=Mn, right of=Cr, Metal] {\NaturalElementTextFormat{25}{54.938}{Mn}{Manganese}{}};
\node[name=Tc, below of=Mn, Metal] {\NaturalElementTextFormat{43}{96}{Tc}{Technetium}{}};
\node[name=Re, below of=Tc, Metal] {\NaturalElementTextFormat{75}{186.21}{Re}{Rhenium}{}};
\node[name=Bh, below of=Re, Metal] {\SyntheticElementTextFormat{107}{264}{Bh}{Bohrium}{}};

%% Group 8 - VIIIB
\node[name=Fe, right of=Mn, Metal] {\NaturalElementTextFormat{26}{55.845}{Fe}{Iron}{}};
\node[name=Ru, below of=Fe, Metal] {\NaturalElementTextFormat{44}{101.07}{Ru}{Ruthenium}{$4d^75s^1$}};
\node[name=Os, below of=Ru, Metal] {\NaturalElementTextFormat{76}{190.23}{Os}{Osmium}{}};
\node[name=Hs, below of=Os, Metal] {\SyntheticElementTextFormat{108}{277}{Hs}{Hassium}{}};

%% Group 9 - VIIIB
\node[name=Co, right of=Fe, Metal] {\NaturalElementTextFormat{27}{58.933}{Co}{Cobalt}{}};
\node[name=Rh, below of=Co, Metal] {\NaturalElementTextFormat{45}{102.91}{Rh}{Rhodium}{$4d^85s^1$}};
\node[name=Ir, below of=Rh, Metal] {\NaturalElementTextFormat{77}{192.22}{Ir}{Iridium}{}};
\node[name=Mt, below of=Ir, Metal] {\SyntheticElementTextFormat{109}{268}{Mt}{Meitnerium}{}};

%% Group 10 - VIIIB
\node[name=Ni, right of=Co, Metal] {\NaturalElementTextFormat{28}{58.693}{Ni}{Nickel}{}};
\node[name=Pd, below of=Ni, Metal] {\NaturalElementTextFormat{46}{106.42}{Pd}{Palladium}{$4d^{10}5s^0$}};
\node[name=Pt, below of=Pd, Metal] {\NaturalElementTextFormat{78}{195.08}{Pt}{Platinum}{$5d^96s^1$}};
\node[name=Ds, below of=Pt, Metal] {\SyntheticElementTextFormat{110}{281}{Ds}{Darmstadtium}{}};

%% Group 11 - IB
\node[name=Cu, right of=Ni, Metal] {\NaturalElementTextFormat{29}{63.546}{Cu}{Copper}{$3d^{10}4s^1$}};
\node[name=Ag, below of=Cu, Metal] {\NaturalElementTextFormat{47}{107.87}{Ag}{Silver}{$4d^{10}5s^1$}};
\node[name=Au, below of=Ag, Metal] {\NaturalElementTextFormat{79}{196.97}{Au}{Gold}{$5d^{10}6s^1$}};
\node[name=Rg, below of=Au, Metal] {\SyntheticElementTextFormat{111}{280}{Rg}{Roentgenium}{}};

%% Group 12 - IIB
\node[name=Zn, right of=Cu, Metal] {\NaturalElementTextFormat{30}{65.39}{Zn}{Zinc}{}};
\node[name=Cd, below of=Zn, Metal] {\NaturalElementTextFormat{48}{112.41}{Cd}{Cadmium}{}};
\node[name=Hg, below of=Cd, Metal] {\NaturalElementTextFormat{80}{200.59}{Hg}{Mercury}{}};
\node[name=Uub, below of=Hg, Metal] {\SyntheticElementTextFormat{112}{285}{Uub}{Ununbium}{}};

%% Lanthanide
\node[name=La, below of=AcLr, LanthanideActinide, yshift=-1cm] {\NaturalElementTextFormat{57}{138.91}{La}{Lanthanum}{$5d^16s^2$}};
\node[name=Ce, right of=La, LanthanideActinide] {\NaturalElementTextFormat{58}{140.12}{Ce}{Cerium}{$4f^15d^16s^2$}};
\node[name=Pr, right of=Ce, LanthanideActinide] {\NaturalElementTextFormat{59}{140.91}{Pr}{Praseodymium}{}};
\node[name=Nd, right of=Pr, LanthanideActinide] {\NaturalElementTextFormat{60}{144.24}{Nd}{Neodymium}{}};
\node[name=Pm, right of=Nd, LanthanideActinide] {\NaturalElementTextFormat{61}{145}{Pm}{Promethium}{}};
\node[name=Sm, right of=Pm, LanthanideActinide] {\NaturalElementTextFormat{62}{150.36}{Sm}{Samarium}{}};
\node[name=Eu, right of=Sm, LanthanideActinide] {\NaturalElementTextFormat{63}{151.96}{Eu}{Europium}{}};
\node[name=Gd, right of=Eu, LanthanideActinide] {\NaturalElementTextFormat{64}{157.25}{Gd}{Gadolinium}{$4f^75d^16s^2$}};
\node[name=Tb, right of=Gd, LanthanideActinide] {\NaturalElementTextFormat{65}{158.93}{Tb}{Terbium}{$4f^85d^16s^2$}};
\node[name=Dy, right of=Tb, LanthanideActinide] {\NaturalElementTextFormat{66}{162.50}{Dy}{Dysprosium}{}};
\node[name=Ho, below of=Eu, LanthanideActinide, yshift=-0.5cm] {\NaturalElementTextFormat{67}{164.93}{Ho}{Holmium}{}};
\node[name=Er, right of=Ho, LanthanideActinide] {\NaturalElementTextFormat{68}{167.26}{Er}{Erbium}{}};
\node[name=Tm, right of=Er, LanthanideActinide] {\NaturalElementTextFormat{69}{168.93}{Tm}{Thulium}{}};
\node[name=Yb, right of=Tm, LanthanideActinide] {\NaturalElementTextFormat{70}{173.04}{Yb}{Ytterbium}{}};
\node[name=Lu, right of=Yb, LanthanideActinide] {\NaturalElementTextFormat{71}{174.97}{Lu}{Lutetium}{}};

%% Actinide
\node[name=Ac, below of=La, LanthanideActinide, yshift=-4.5cm] {\NaturalElementTextFormat{89}{227}{Ac}{Actinium}{$6d^17s^2$}};
\node[name=Th, right of=Ac, LanthanideActinide] {\NaturalElementTextFormat{90}{232.04}{Th}{Thorium}{$6d^27s^2$}};
\node[name=Pa, right of=Th, LanthanideActinide] {\NaturalElementTextFormat{91}{231.04}{Pa}{Protactinium}{$5f^26d^17s^2$}};
\node[name=U, right of=Pa, LanthanideActinide] {\NaturalElementTextFormat{92}{238.03}{U}{Uranium}{$5f^36d^17s^2$}};
\node[name=Np, right of=U, LanthanideActinide] {\SyntheticElementTextFormat{93}{237}{Np}{Neptunium}{$5f^46d^17s^2$}};
\node[name=Pu, right of=Np, LanthanideActinide] {\SyntheticElementTextFormat{94}{244}{Pu}{Plutonium}{}};
\node[name=Am, right of=Pu, LanthanideActinide] {\SyntheticElementTextFormat{95}{243}{Am}{Americium}{}};
\node[name=Cm, right of=Am, LanthanideActinide] {\SyntheticElementTextFormat{96}{247}{Cm}{Curium}{}};
\node[name=Bk, right of=Cm, LanthanideActinide] {\SyntheticElementTextFormat{97}{247}{Bk}{Berkelium}{$5f^76d^17s^2$}};
\node[name=Cf, right of=Bk, LanthanideActinide] {\SyntheticElementTextFormat{98}{251}{Cf}{Californium}{}};
\node[name=Es, below of=Am, LanthanideActinide, yshift=-0.5cm] {\SyntheticElementTextFormat{99}{252}{Es}{Einsteinium}{}};
\node[name=Fm, right of=Es, LanthanideActinide] {\SyntheticElementTextFormat{100}{257}{Fm}{Fermium}{}};
\node[name=Md, right of=Fm, LanthanideActinide] {\SyntheticElementTextFormat{101}{258}{Md}{Mendelevium}{}};
\node[name=No, right of=Md, LanthanideActinide] {\SyntheticElementTextFormat{102}{259}{No}{Nobelium}{}};
\node[name=Lr, right of=No, LanthanideActinide] {\SyntheticElementTextFormat{103}{262}{Lr}{Lawrencium}{}};

%% Period
\node[name=Period1, left of=H, PeriodLabel] {1};
\node[name=Period2, left of=Li, PeriodLabel] {2};
\node[name=Period3, left of=Na, PeriodLabel] {3}; 
\node[name=Period4, left of=K, PeriodLabel] {4}; 
\node[name=Period4i, left of=Sc, PeriodLabel] {4}; 
\node[name=Period5, left of=Rb, PeriodLabel] {5};
\node[name=Period5i, left of=Y, PeriodLabel] {5};
\node[name=Period6, left of=Cs, PeriodLabel] {6};
\node[name=Period6i, left of=LaLu, PeriodLabel] {6};
\node[name=Period6ii, left of=La, PeriodLabel] {6};
\node[name=Period7, left of=Fr, PeriodLabel] {7};
\node[name=Period7i, left of=AcLr, PeriodLabel] {7};
\node[name=Period7ii, left of=Ac, PeriodLabel] {7};

%% Group
\node[name=Group1, above of=H, GroupLabel] {1 \hfill IA};
\node[name=Group2, above of=Be, GroupLabel] {2 \hfill IIA};
\node[name=Group3, above of=Sc, GroupLabel] {3 \hfill IIIA};
\node[name=Group4, above of=Ti, GroupLabel] {4 \hfill IVB};
\node[name=Group5, above of=V, GroupLabel] {5 \hfill VB};
\node[name=Group6, above of=Cr, GroupLabel] {6 \hfill VIB};
\node[name=Group7, above of=Mn, GroupLabel] {7 \hfill VIIB};
\node[name=Group8, above of=Fe, GroupLabel] {8 \hfill VIIIB};
\node[name=Group9, above of=Co, GroupLabel] {9 \hfill VIIIB};
\node[name=Group10, above of=Ni, GroupLabel] {10 \hfill VIIIB};
\node[name=Group11, above of=Cu, GroupLabel] {11 \hfill IB};
\node[name=Group12, above of=Zn, GroupLabel] {12 \hfill IIB};
\node[name=Group13, above of=B, GroupLabel] {13 \hfill IIIA};
\node[name=Group14, above of=C, GroupLabel] {14 \hfill IVA};
\node[name=Group15, above of=N, GroupLabel] {15 \hfill VA};
\node[name=Group16, above of=O, GroupLabel] {16 \hfill VIA};
\node[name=Group17, above of=F, GroupLabel] {17 \hfill VIIA};
\node[name=Group18, above of=He, GroupLabel] {18 \hfill VIIIA};

	%% Draw dotted lines connectin group3-12 breakout to main table
	%  \draw (ScZn.north west) edge[dotted] (Sc.north west)
%        (ScZn.north east) edge[dotted] (Zn.north east)    
	%        (AcUub.south west) edge[dotted] (Ac.south west);
	%        (AcUub.south east) edge[dotted] (Uub.south east);
	%% Draw dotted lines connecting Lanthanide breakout to Group 3-12
	%  \draw (LaLu.north west) edge[dotted] (La.north west)
%        (LaLu.north east) edge[dotted] (Lu.north east)
	%        (LaLu.south west) edge[dotted] (La.south west);
	%        (LaLu.south east) edge[dotted] (Lu.south east);
	%% Draw dotted lines connecting Actinide breakout to Group 3-12
	%  \draw (AcLr.north west) edge[dotted] (Ac.north west)
%        (AcLr.north east) edge[dotted] (Lr.north east)
	%        (AcLr.south west) edge[dotted] (Ac.south west);
	%        (AcLr.south east) edge[dotted] (Lr.south east);

	%% Legend
\draw[black, AlkaliMetalFill] ($(He.east) + (1em,-0.0em)$)
	rectangle +(1em, 1em) node[right, yshift=-1ex]{Alkali Metal};
\draw[black, AlkalineEarthMetalFill] ($(He.east) + (1em,-1.5em)$)
	rectangle +(1em, 1em) node[right, yshift=-1ex]{Alkaline Earth Metal};
\draw[black, MetalFill] ($(He.east) + (1em,-3.0em)$)
	rectangle +(1em, 1em) node[right, yshift=-1ex]{Metal};
\draw[black, MetalloidFill] ($(He.east) + (1em,-4.5em)$)
	rectangle +(1em, 1em) node[right, yshift=-1ex]{Metalloid};
\draw[black, NonmetalFill] ($(He.east) + (1em,-6.0em)$)
	rectangle +(1em, 1em) node[right, yshift=-1ex]{Non-metal};
\draw[black, HalogenFill] ($(He.east) + (1em,-7.5em)$)
	rectangle +(1em, 1em) node[right, yshift=-1ex]{Halogen};
\draw[black, NobleGasFill] ($(He.east) + (1em,-9.0em)$)
	rectangle +(1em, 1em) node[right, yshift=-1ex]{Noble Gas};
\draw[black, LanthanideActinideFill] ($(He.east) + (1em,-10.5em)$)
	rectangle +(1em, 1em) node[right, yshift=-1ex]{Lanthanide/Actinide};

\node at ($(He.east) + (5em,-15em)$) [name=elementLegend, Element, fill=white]
{\NaturalElementTextFormat{Z}{mass}{Symbol}{Name}{}};
\node[Element, fill=white, right of=elementLegend, xshift=1em]
{\SyntheticElementTextFormat{}{}{man-made}{}{}} ;

%% Diagram Title
%  \node at (H.west -| Fe.north) [name=diagramTitle, TitleLabel]
%    {(Mendeleev's) Periodic Table of Chemical Elements via Ti\emph{k}Z};

\end{tikzpicture}


\chapter{Naturkonstanten}
\newcommand{\nat}[3]{& $#1$ & $\SI{#2}{#3}$}
\begin{tabular}{lr@{= }l}
absoluter Nullpunk \nat{T_0}{-273.1}{\celsius} \\
		  Radius Atom \nat{r_{Atom}}{1}{\angstrom} $= \SI{1e-10}{\metre}$ \\
		  Radius Proton \nat{r_{Proton}}{1.5e-15}{\metre} \\
		  Masse Elektron \nat{m_{El}}{9.109e-31}{\kilogram} \\
		  g-Faktor \nat{g_{El}}{2.002}{} \\
		  Masse Proton \nat{m_{Pr}}{1.673e-27}{\kilogram} \\
		  g-Faktor \nat{g_{Pr}}{5.586}{} \\
		  Elementarladung \nat{e}{1.602e-19}{\coulomb} \\
		  Rydberg-Energie \nat{R}{13.6}{\electronvolt} \\
		  Bohrsches Magnetron (Elektron) \nat{\mu_B}{5.78e-5}{\electronvolt \per \tesla} \\
		  \nat{}{9.27e-24}{\joule \per \tesla}\\
		  Gravitationskonstante \nat{\gamma}{6.674e-11}{\cubic \metre \per \kilo \gram \per \square \second} \\
		  Erdgravitation \nat{g_{Erde}}{9.81}{\metre \per \square \second} \\
		  Magnetische Feldonstante \nat{\mu_0}{4\pi e-7}{\newton \per \square \ampere} \\
		  Elektrische Feldkonstante \nat{\epsilon_0}{8.8542e-12}{\ampere \second \per \volt \per \metre} \\
		  Wirkungsquantum \nat{h}{6.626e-34}{\joule \second}\\
		  \nat{}{4.136e-15}{\electronvolt}\\
		  \nat{\hbar}{1.054e-34}{\joule \second}\\
		  \nat{}{6.582e-16}{\electronvolt}\\
		  Boltzmann Konstante \nat{k_B}{1.38e-23}{\joule \per \kelvin}\\
		  \nat{}{8.62e-5}{\electronvolt \per \kelvin}\\
		  universelle Gaskonstante \nat{R}{8.31}{\joule \per \mol \per \kelvin}\\
		  Avogadro-Konstante \nat{N_A}{6.022e23}{\per \mol}
	  \end{tabular}




\end{document}
